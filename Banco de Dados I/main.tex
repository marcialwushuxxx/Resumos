\documentclass[12pt]{article}
\usepackage{graphicx,url}
\usepackage[brazil]{babel}   
%\usepackage[latin1]{inputenc}  
\usepackage[utf8]{inputenc}  %permite texto com acento
% UTF-8 encoding is recommended by ShareLaTex
\usepackage{verbatim}
\usepackage{listings}
\usepackage{xcolor}
\usepackage[margin=2.5cm]{geometry}
\usepackage{xcolor}
\usepackage{fancyhdr}
\pagestyle{fancy}
\fancyhf{}
\rfoot{Página \thepage}
\lhead{\it{\textcolor{gray}{\small II Mostra Científica de Tecnologia - FGF}}}
\rhead{\it{\textcolor{gray}{\small 01 e 03 de Março de 2017}}}
\chead{\it{\textcolor{gray}{\small Fortaleza - Ce}}}
\lfoot{\it{X Feira Tecnológica - FGF}}
\cfoot{\it{Fortaleza - Ceará}}
\renewcommand{\headrulewidth}{0.5pt}

%%%%%%%%%%%%%%%%%%%%%%%%%%%%

\begin{document}
%\onehalfspacing

%
%%% Não alterar o preâmbulo acima.

    %%% MARQUE UM X NO TIPO DE PESQUISA

\noindent{TIPO DE RESUMO: 1. Trabalho original( ), 2. Relato de experiência( ), 3.Estudo de caso( ), 4. Pesquisa bibliométrica(X), 5. Reflexão crítica( ), 6. Relatório técnico ( ), 7. Trabalho de conclusão de curso( ), 8. Nota prévia de monografia( ), 9. Relatório final de monografia( ).\\ASSINALE O TIPO DE APRESENTAÇÃO: 1. ORAL(X) 2. POSTER( ).
}
%%%%%%%%%%% TÍTULO %%%%%%%%%%%

\begin{center}
\textbf{\Large{SISTEMAS DE BANCO DE DADOS PARA A INTERNET DAS COISAS}}\\
\end{center}

\vspace*{0.2cm}
%%%%%%%%%%% AUTORES - NO MÁXIMO 2 ALUNOS E 1 ORIENTADOR - %%%%%%%%%%%

\begin{flushright}
 {\bf Cleilson de Sousa Pereira, Jefferson de Almeida Guimarães} \footnote[1]{Graduando em Sistemas para Internet - FGF. e-mail: \it cleilsonpereira@aluno.fgf.edu.br\hbox      | jefferson@aluno.fgf.edu.br}  \\
  {\bf Eder de Sousa da Silva, Claudia Adrielle Diogo Mesquita} \footnote[2]{Graduando em Sistemas para Internet - FGF. e-mail: \it ederss@aluno.fgf.edu.br\hbox |
  adriellediogo@aluno.fgf.edu.br}  \\
  {\bf Adriana Maria Rebouças do Nascimento} \footnote[3]{Mestre - Faculdade da Grande Fortaleza. e-mail: \it adri@fgf.edu.br}   \\
\end{flushright}

\vspace*{0.5cm}

%%%%%%%%%%% CORPO DO TRABALHO - MAXIMO DE 500 PALAVRAS %%%%%%%%%%%

\noindent{\textbf{INTRODUÇÃO:} A revolução do paradigma da Internet das Coisas, uma tradução livre para Internet of Things (IoT),com o intuito de conectar dispositivos eletrônicos do dia-a-dia à internet, vem inovando e crescendo com o tempo, por meio da inovação tecnológica nos campos de sensores wireless, inteligência artificial e nanotecnologia. E essa nova onda tecnológica, muda o cenário dos sistemas de informação, tendo que se adaptar ao novo conceito de fluxo de dados, exigindo uma inovação nos sistemas gerenciadores de banco de dados com a Internet de todas as coisas atuando em diversas plataformas, e estimando um fluxo de dados maior do que hoje chamamos de big data.\textbf{OBJETIVO:} O objetivo desse trabalho é analisar a produção cientifica ao respeito do desenvolvimento de novos modelos de SGBD para o uso da internet das coisas, e listar as principais soluções encontradas no mercado para IoT, apresentando estatisticas quantitativas sobre o assunto. \textbf{METODOLOGIA:} A metodologia aplicada nesse seguinte trabalho é a pesquisa bibliométrica, que é um campo das áreas de biblioteconomia e ciência da informação, que aplica metodos estatísticos e matemáticos para analisar e construir indicadores sobre a evolução cientifica e sua publicação em periódicos de pesquisas e publicações cientificas. O modelo de aplicação do uso da bibliométria nesse artigo é a tendência e crescimento de publicações acerca do assunto exposto no titulo dele. \textbf{RESULTADOS E DISCUSSÃO:} A analise resultou uma crescente produção de artigos, no periodo de pesquisa entre os anos de 2000 a 2017, foram pesquisados nos principais periodicos cientificos, Capes, Spell, Scielo Google Acadêmico e Academic Microsoft. A modelagem dos dados foi elaborada utilizando Tableau Pubilc e R, gerando graficos de visualização estatistica. \textbf{CONCLUSÃO:} A conclusão com base nos dados coletados durante esse periodo, foi a elaboração de novas interconexões tecnológicas, e a presente iniciativa de desenvolvimento de meios para garantir a coleta e processamento de dados nesse paradigma modificador da industria. \textbf{REFERÊNCIAS BIBLIOGRÁFICAS:} MATOS, D. Database of Things (DoT) – Banco de Dados das Coisas. DAL BIANCO, G. Banco de Dados em Memória sobre Clusters de Computadores. PIRES, P. DELICATO, F. BATISTA, T. BARROS, T. CAVALCANTE, E. Plataformas para a Internet das Coisas}

%%%%%%%%%%% FIM DO RESUMO %%%%%%%%%%%

\end{document}


