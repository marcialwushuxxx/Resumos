\documentclass[12pt]{article}
\usepackage{graphicx,url}
\usepackage[brazil]{babel}   
%\usepackage[latin1]{inputenc}  
\usepackage[utf8]{inputenc}  %permite texto com acento
% UTF-8 encoding is recommended by ShareLaTex
\usepackage{verbatim}
\usepackage{listings}
\usepackage{xcolor}
\usepackage[margin=2.5cm]{geometry}
\usepackage{xcolor}
\usepackage{fancyhdr}
\pagestyle{fancy}
\fancyhf{}
\rfoot{Página \thepage}
\lhead{\it{\textcolor{gray}{\small II Mostra Científica de Tecnologia - FGF}}}
\rhead{\it{\textcolor{gray}{\small 02, 03 e 04 de Março de 2017}}}
\chead{\it{\textcolor{gray}{\small Fortaleza - Ce}}}
\lfoot{\it{X Feira Tecnológica - FGF}}
\cfoot{\it{Fortaleza - Ceará}}
\renewcommand{\headrulewidth}{0.5pt}

%%%%%%%%%%%%%%%%%%%%%%%%%%%%

\begin{document}
%\onehalfspacing

%
%%% Não alterar o preâmbulo acima.

    %%% MARQUE UM X NO TIPO DE PESQUISA

\noindent{TIPO DE RESUMO: 1. Trabalho original(X), 2. Relato de experiência( ), 3.Estudo de caso( ), 4. Pesquisa bibliométrica( ), 5. Pesquisa bibliográfica ( ), 6. Reflexão crítica( ), 7. Relatório técnico ( ), 8. Trabalho de conclusão de curso( ), 9. Nota prévia de monografia( ), 10. Relatório final de monografia( ).\\ASSINALE O TIPO DE APRESENTAÇÃO: 1. ORAL(X) 2. POSTER( ).
}
%%%%%%%%%%% TÍTULO %%%%%%%%%%%

\begin{center}
\textbf{\Large{LEVANTAMENTO DE INFORMAÇÕES SOBRE AGENDAMENTO ONLINE PARA SISTEMAS DOS POSTOS DE SAÚDE EM FORTALEZA E REGIÃO METROPOLITANA}}\\
\end{center}

\vspace*{0.2cm}
%%%%%%%%%%% AUTORES - NO MÁXIMO 4 E 1 ORIENTADOR - %%%%%%%%%%%

\begin{flushright}
 {\bf Cleilson de Sousa Pereira, Jefferson de Almeida Guimarães} \footnote[1]{Graduando em Sistemas para Internet - FGF. e-mail: \it cleilsonpereira@aluno.fgf.edu.br\hbox      | jefferson@aluno.fgf.edu.br}  \\
  {\bf Eder de Sousa da Silva, Claudia Adrielle Diogo Mesquita} \footnote[2]{Graduando em Sistemas para Internet - FGF. e-mail: \it ederss@aluno.fgf.edu.br\hbox |
  adriellediogo@aluno.fgf.edu.br}  \\
  {\bf Marcos Venícius Mourão Araújo} \footnote[3]{Especialista - Faculdade da Grande Fortaleza. e-mail: \it marcos@fgf.edu.br}   \\
\end{flushright}

\vspace*{0.5cm}

%%%%%%%%%%% CORPO DO TRABALHO - ENTRE 200 E 600 PALAVRAS %%%%%%%%%%%
%%%% TODOS OS TRABALHOS DEVEM TRAZER AS SEÇÕES EXATAMENTE COMO DESTACADO ABAIXO %%%%%%


\noindent{\textbf{INTRODUÇÃO:} O sistema único de saúde nos postos de saúde em Fortaleza trabalham com o UniSUS Web para fazer todo cadastro, agendamento, consulta e encaminhamento para redes médicas conveniadas. Implantado pelo governo do estado do Ceará em 2014 em toda rede pública de saúde. O serviço de agendamento de consulta é feito via \textit{web}, mas, no posto de saude e por um funcionário treinado. \textbf{OBJETIVO:} Analisar atraves de estudos as soluções de agendamento e a viabilidade conceitual do projeto para o  funcionamento na estrutura de atendimento a saúde primária, apresentando um prototipo conceitual de layout e lógico codificado. Utilizando como metodologia de desenvolvimento a prototipagem para entregar um sistema/modelo sem funcionalidade inteligentes. O presente trabalho visa apresentar uma solução complementar para o cidadão, um serviços web de agendamento online para o Serviço de Marcação de Consulta e Exames (SAME). \textbf{METODOLOGIA:} Realizado um levantamento de informações, analisando os serviços web disponiveis nos portais da secretarias municipais do estado do Ceará, identificado como um problema direcionado para nosso estudo, a ausência de um serviço \textit{web} para o usuário realizar seu agendamento na unidade de assitência médica primária. \textbf{RESULTADOS E DISCUSSÃO:} Foi pesquisado nos portais na secretárias municiapais de Fortaleza e região metropolitana(Maracanaú, Maranguape,Eusébio, Caucaia), nenhum serviço de agendamento \textit{online} direcionamento para a área da saúde foi encontrado. O modelo de referência encontrado foi na secretária de São Paulo, onde o serviço de saúde é feito por agendamento \textit{online} e o Sistema Nacional de Saúde de Portugal também trabalha com esse modelo. \textbf{CONCLUSÃO:} A viabilidade do desenvolvimento do prototipo foi considerada positiva, pois de acordo com as pesquisas casos do uso desse tipo de serviço em redes privada de assistência médica já existem. \textbf{REFERÊNCIAS BIBLIOGRÁFICAS (MAX. TRÊS):} SP, PREFEITURA. Manual agendamento eletronico. DISPONIVEL http://agendamentosf.prefeitura.sp.gov.br/Passo_Passo_Internet_versao1 5.pdf.}
%%%%%%%%%%% FIM DO RESUMO %%%%%%%%%%%
%%%%%%%%%%% O RESUMO DEVE TER NO MÁXIMO DUAS PÁGINAS %%%%%%%%%%%


\end{document}


