\documentclass[12pt]{article}
\usepackage{graphicx,url}
\usepackage[brazil]{babel}   
%\usepackage[latin1]{inputenc}  
\usepackage[utf8]{inputenc}  %permite texto com acento
% UTF-8 encoding is recommended by ShareLaTex
\usepackage{verbatim}
\usepackage{listings}
\usepackage{xcolor}
\usepackage[margin=2.5cm]{geometry}
\usepackage{xcolor}
\usepackage{fancyhdr}
\pagestyle{fancy}
\fancyhf{}
\rfoot{Página \thepage}
\lhead{\it{\textcolor{gray}{\small II Mostra Científica de Tecnologia - FGF}}}
\rhead{\it{\textcolor{gray}{\small 02, 03 e 04 de Março de 2017}}}
\chead{\it{\textcolor{gray}{\small Fortaleza - Ce}}}
\lfoot{\it{X Feira Tecnológica - FGF}}
\cfoot{\it{Fortaleza - Ceará}}
\renewcommand{\headrulewidth}{0.5pt}

%%%%%%%%%%%%%%%%%%%%%%%%%%%%

\begin{document}
%\onehalfspacing

%
%%% Não alterar o preâmbulo acima.

    %%% MARQUE UM X NO TIPO DE PESQUISA

\noindent{TIPO DE RESUMO: 1. Trabalho original( ), 2. Relato de experiência( ), 3.Estudo de caso( ), 4. Pesquisa bibliométrica(X), 5. Pesquisa bibliográfica ( ), 6. Reflexão crítica( ), 7. Relatório técnico ( ), 8. Trabalho de conclusão de curso( ), 9. Nota prévia de monografia( ), 10. Relatório final de monografia( ).\\ASSINALE O TIPO DE APRESENTAÇÃO: 1. ORAL(X) 2. POSTER( ).
}
%%%%%%%%%%% TÍTULO %%%%%%%%%%%

\begin{center}
\textbf{\Large{UMA ANALISE BIBLIOMÉTRICA SOBRE A GERÊNCIA DE PROJETOS NA ÁREA DA SAÚDE PÚBLICA}}\\
\end{center}

\vspace*{0.2cm}
%%%%%%%%%%% AUTORES - NO MÁXIMO 4 E 1 ORIENTADOR - %%%%%%%%%%%

\begin{flushright}
 {\bf Cleilson de Sousa Pereira} \footnote[1]{Graduando em Sistemas para Internet - FGF. e-mail: \it cleilsonpereira@aluno.fgf.edu.br}  \\
 {\bf José Henrique Girão Nogueira} \footnote[2]{Graduando em Sistema para Internet - FGF. e-mail: \it henriquenogueira45@gmail.com}  \\
  {\bf Madson Dantas Lima} \footnote[3]{Graduando em Ciência da Computação - FGF. e-mail: \it madson109@aluno.fgf.edu.br}  \\
   {\bf Elidiane Martins Freitas} \footnote[4]{Mestre - Faculdade da Grande Fortaleza. e-mail: \it elidiane@fgf.edu.br}   \\
\end{flushright}

\vspace*{0.5cm}

%%%%%%%%%%% CORPO DO TRABALHO - ENTRE 200 E 600 PALAVRAS %%%%%%%%%%%
%%%% TODOS OS TRABALHOS DEVEM TRAZER AS SEÇÕES EXATAMENTE COMO DESTACADO ABAIXO %%%%%%


\noindent{\textbf{INTRODUÇÃO:} O presente artigo vem abordar a incorporação da Gerência de Projetos direcionado à área da Saúde, usado como uma ferramenta de auxilio a Gestão de Qualidade do serviço hospitalar,além do \textit{Benchmarking} utitlizado, a criação de projetos para estruturar processos, tem auxiliado gestores a definir por meio das normas do PMI da Gerência de Projetos, direcionamentos organizacionais. \textbf{OBJETIVO:}O objetivo visa mensurar e quantificar através de pesquisas a produçaõ acadêmica na publicação cientifica, observando o direcionamento estratégico de gestão dada ao sistema de serviço hospitalar, se utilizando das normas e diretrizes do PMBOK. \textbf{METODOLOGIA:} A metodologia de pesquisa aplicada ao presente trabalho foi a pesquisa bibliométrica, uma area da ciência da informação para realizar uma analise e quantificar os dados de produção acadêmica sobre um determinado tema, pesquisado nos periodicos Capes, Spell, Scielo, Goolgle Acadêmico e Academic Microsoft; modelagem dos dados para visualização grafica, utilizado as ferramentas Tableau Pubic e R. \textbf{RESULTADOS E DISCUSSÃO:} Realizado pesquisa utilizando palavras chaves em português, feito uma analise geral com termos genéricos e depois uma pesquisa mais especifica para área de conhecimento em conjunto com as palavras chaves usuais. \textbf{CONCLUSÃO:} Os dados coletados na pesquisa, direciona o envolvimento e preocupação em meios de Gerenciamento e controle da Qualidade dos serviços prestados em unidade hospitalares, tendo como base de melhoria estratégica na Gestão o uso do Gerenciamento de Projetos, apesar da dificuldade formulada por alguns autores, devido a complexidade do sistema hospitalar. Certficação de Acreditação Hospitalar é o foco gerencial em alguns artigos externos analisados. \textbf{REFERÊNCIAS BIBLIOGRÁFICAS :}MARIA RAMOS FREIRE, E. CRISTINA ROCHA BATISTA, R. REGINA MARTINEZ, M. Gerenciamento de projetos voltado para acreditação hospitalar: 
estudo de caso. MARIA RAMOS FREIRE, E. REGINA MARTINEZ, M. Gerenciamento de Projeto como ferramenta de auxilio em Gestão da Qualidade Hospitalar: Um novo campo de atuação profissional para enfermagem}

%%%%%%%%%%% FIM DO RESUMO %%%%%%%%%%%
%%%%%%%%%%% O RESUMO DEVE TER NO  DUAS PÁGINAS %%%%%%%%%%%


\end{document}


