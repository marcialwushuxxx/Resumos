\documentclass[12pt]{article}
\usepackage{graphicx,url}
\usepackage[brazil]{babel}   
%\usepackage[latin1]{inputenc}  
\usepackage[utf8]{inputenc}  %permite texto com acento
% UTF-8 encoding is recommended by ShareLaTex
\usepackage{verbatim}
\usepackage{listings}
\usepackage{xcolor}
\usepackage[margin=2.5cm]{geometry}
\usepackage{xcolor}
\usepackage{fancyhdr}
\pagestyle{fancy}
\fancyhf{}
\rfoot{Página \thepage}
\lhead{\it{\textcolor{gray}{\small II Mostra Científica de Tecnologia - FGF}}}
\rhead{\it{\textcolor{gray}{\small 02, 03 e 04 de Março de 2017}}}
\chead{\it{\textcolor{gray}{\small Fortaleza - Ce}}}
\lfoot{\it{X Feira Tecnológica - FGF}}
\cfoot{\it{Fortaleza - Ceará}}
\renewcommand{\headrulewidth}{0.5pt}

%%%%%%%%%%%%%%%%%%%%%%%%%%%%

\begin{document}
%\onehalfspacing

%
%%% Não alterar o preâmbulo acima.

    %%% MARQUE UM X NO TIPO DE PESQUISA

\noindent{TIPO DE RESUMO: 1. Trabalho original( ), 2. Relato de experiência( ), 3.Estudo de caso( ), 4. Pesquisa bibliométrica(X), 5. Pesquisa bibliográfica ( ), 6. Reflexão crítica( ), 7. Relatório técnico ( ), 8. Trabalho de conclusão de curso( ), 9. Nota prévia de monografia( ), 10. Relatório final de monografia( ).\\ASSINALE O TIPO DE APRESENTAÇÃO: 1. ORAL(X) 2. POSTER( ).
}
%%%%%%%%%%% TÍTULO %%%%%%%%%%%

\begin{center}
\textbf{\Large{UM ESTUDO BIBLIOMÉTRICO SOBRE A DEEP WEB, DARKNET E SUAS REDES ALTERNATIVAS}}\\
\end{center}

\vspace*{0.2cm}
%%%%%%%%%%% AUTORES - NO MÁXIMO 4 E 1 ORIENTADOR - %%%%%%%%%%%

\begin{flushright}
 {\bf Cleilson de Sousa Pereira} \footnote[1]{Graduando em Sistemas para Internet - FGF. e-mail: \it cleilsonpereira@aluno.fgf.edu.br}  \\
 {\bf Thyago Denilson Barbosa da Silva} \footnote[2]{Graduando em Sistemas para Internet - FGF. e-mail: \it aluno@aluno.fgf.edu.br}  \\
  {\bf Madson Dantas Lima} \footnote[3]{Graduando em Ciência da Computação - FGF. e-mail: \it madson109@aluno.fgf.edu.br}  \\
   {\bf Rafael Teixeira de Araujo} \footnote[4]{Mestre Doutorando- Faculdade da Grande Fortaleza. e-mail: \it rafaelteixeira@fgf.edu.br}   \\
\end{flushright}

\vspace*{0.5cm}

%%%%%%%%%%% CORPO DO TRABALHO - ENTRE 200 E 600 PALAVRAS %%%%%%%%%%%
%%%% TODOS OS TRABALHOS DEVEM TRAZER AS SEÇÕES EXATAMENTE COMO DESTACADO ABAIXO %%%%%%


\noindent{\textbf{INTRODUÇÃO:} Deep Web, Darknet e Invisible Web são alguns termos usados nos ultimos anos. A maioria dos artigos jornalisticos mencionam o termo darknet e deep web associado a conteudo criminoso e lendas urbanas da internet. A rede mais mencionada é Rede TOR (The Onion Routing), mas a rede TOR é apenas um percentual da chamada Darknet(Deep Web). Redes alternativas e anônimas iguais ao Projeto TOR, ou similares, estão presente na internet, e as vezes são esquecidas ao se falar se Deep Web(Darknet).  \textbf{OBJETIVO:}Analisar o conteudo produzido academicamente acerca da Deep Web e averiguar trabalhos sobre a chamada Darknet e quais Redes Alternativas são abordadas.Gerando um levantamento estatisco do conteudo especificado. \textbf{METODOLOGIA:} Pesquisa Bibliométrica, realizando um levantamento quantitativo da produção cientifica. Verificado os periodicos Capes, Scielo, Spell, Google Acadêmico, Academic Microsoft, Springer Link, ACM Digital Library e ScienceDirect. Modelagem dos dados com as ferramentas Tableau Public e R Studio. \textbf{RESULTADOS E DISCUSSÃO:} Ecnotrado um grande volume de publicações no periodico do Google, nos demais, resultados alternando entre 900 a 0. Ao expeciifcar a palavra-chave com um nome exato de uma rede anônima de conexão a internet,encontra-se valores de percentual de publicação médio para as principais redes de conhecimentos geral como TOR, Freenet e I2P, para a demais redes pouco conhecidas resultados minimos. \textbf{CONCLUSÃO:} Grande parte dos resultados voltado a Darknet, é feita menção apenas a Rede TOR, as demais redes que compõem essa fração de conteudo não indexado na internet, é encontrada referindo-se ao nome exato da Rede, ou em alguns casos não apresenta um retorno. Não encontrado em periodicos uma catalogação dessas Redes Alternativas. \textbf{REFERÊNCIAS BIBLIOGRÁFICAS (MAX. TRÊS):} BERGMAN, M. The Deep Web: surfacing hidden value. BYRNE, J. KIMBALL, K. Inside the Darknet: Techno-Crime And Criminal Opportunity}

%%%%%%%%%%% FIM DO RESUMO %%%%%%%%%%%
%%%%%%%%%%% O RESUMO DEVE TER NO MÁXIMO DUAS PÁGINAS %%%%%%%%%%%


\end{document}


