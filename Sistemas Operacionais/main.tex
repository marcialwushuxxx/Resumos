\documentclass[12pt]{article}
\usepackage{graphicx,url}
\usepackage[brazil]{babel}   
%\usepackage[latin1]{inputenc}  
\usepackage[utf8]{inputenc}  %permite texto com acento
% UTF-8 encoding is recommended by ShareLaTex
\usepackage{verbatim}
\usepackage{listings}
\usepackage{xcolor}
\usepackage[margin=2.5cm]{geometry}
\usepackage{xcolor}
\usepackage{fancyhdr}
\pagestyle{fancy}
\fancyhf{}
\rfoot{Página \thepage}
\lhead{\it{\textcolor{gray}{\small II Mostra Científica de Tecnologia - FGF}}}
\rhead{\it{\textcolor{gray}{\small 02, 03 e 04 de Março de 2017}}}
\chead{\it{\textcolor{gray}{\small Fortaleza - Ce}}}
\lfoot{\it{X Feira Tecnológica - FGF}}
\cfoot{\it{Fortaleza - Ceará}}
\renewcommand{\headrulewidth}{0.5pt}

%%%%%%%%%%%%%%%%%%%%%%%%%%%%

\begin{document}
%\onehalfspacing

%
%%% Não alterar o preâmbulo acima.

    %%% MARQUE UM X NO TIPO DE PESQUISA

\noindent{TIPO DE RESUMO: 1. Trabalho original( ), 2. Relato de experiência( ), 3.Estudo de caso( ), 4. Pesquisa bibliométrica(X), 5. Pesquisa bibliográfica ( ), 6. Reflexão crítica( ), 7. Relatório técnico ( ), 8. Trabalho de conclusão de curso( ), 9. Nota prévia de monografia( ), 10. Relatório final de monografia( ).\\ASSINALE O TIPO DE APRESENTAÇÃO: 1. ORAL(X) 2. POSTER( ).
}
%%%%%%%%%%% TÍTULO %%%%%%%%%%%

\begin{center}
\textbf{\Large{SISTEMAS EMBARCADOS NA AUTOMAÇÃO INDUSTRIAL}}\\
\end{center}

\vspace*{0.2cm}
%%%%%%%%%%% AUTORES - NO MÁXIMO 4 E 1 ORIENTADOR - %%%%%%%%%%%

\begin{flushright}
 {\bf Cleilson de Sousa Pereira, Jefferson de Almeida Guimarães} \footnote[1]{Graduando em Sistemas para Internet - FGF. e-mail: \it cleilsonpereira@aluno.fgf.edu.br\hbox      | jefferson@aluno.fgf.edu.br}  \\
  {\bf Eder de Sousa da Silva, Tharles Michael Batista Amaro} \footnote[2]{Graduando em Sistemas para Internet - FGF. e-mail: \it ederss@aluno.fgf.edu.br\hbox |
  tharlesamaro@aluno.fgf.edu.br}  \\
  {\bf Francisco das Chagas de Carvalho Junior} \footnote[3]{Mestre - Faculdade da Grande Fortaleza. e-mail: \it chagas@fgf.edu.br}   \\
\end{flushright}

\vspace*{0.5cm}

%%%%%%%%%%% CORPO DO TRABALHO - ENTRE 200 E 600 PALAVRAS %%%%%%%%%%%
%%%% TODOS OS TRABALHOS DEVEM TRAZER AS SEÇÕES EXATAMENTE COMO DESTACADO ABAIXO %%%%%%


\noindent{\textbf{INTRODUÇÃO:} Sistema embarcado é um sistema microprocessado, no qual o computador é encapsulado ou dedicado ao dispositivo ou sistema que ele controla. Um sistema embarcado realiza um conjunto de tarefas predefinidas, geralmente com requisitos especificos, o software escrito para para sitemas embarcados é chamado de firmware e armazendo em memória ROM ou flash em vez de disco rigido, e executado com recursos limitados. A automação industrial em diversos segmentos, com o intuito de agilizar e maximizar a produção e melhorar a qualidade no processo, tem utilizado esses sistemas microprocessados em grande escala nos últimos anos, podendo realizar a substituição da mão de obra humana em certos processos a um custo beneficio extremamente baixo; fabricantes de diversos setores estão embutindo softwares em seus produtos, desde a área autobilistica à dispositivos médicos. \textbf{OBJETIVO:}Obter uma visão da produção cientifica acerca da automação industrial utilizando sistemas embarcados, apesar da automação industrial ter registro desde os anos 60, na chamada terceira revolução industrial, será utilizado como linha de tempo de pesquisa os anos de 2000 a 2017. \textbf{METODOLOGIA:} Com objetivo de retratar o grau de desenvolvimento dessa área de conhecimento, utilizaremos a pesquisa bibliométrica para análise da produção cientica e acadêmica, pesquisando nos principais periódicos cientificos, Capes, Scielo, Spell e Google Academico. Modelagem dos dados graficos serão realizados utilizando as ferramentas Tableau Public e R. \textbf{RESULTADOS E DISCUSSÃO:} Encontrado uma demanda de produção cientifica como forte ascendência nessa ultima decada visando os estudo e desenvolvimento de integração e automatização de servços nos mais variados segmentos industriais. \textbf{CONCLUSÃO:} O presente artigo analisou a demanda de produção cientifica e as principais soluções de sistemas ambarcados utilizados, encontramos uma tendência crescente para novos artigos complementares abordando a industria 4.0 e a internet industrial complementando a integração com o paradigma IoT. \textbf{REFERÊNCIAS BIBLIOGRÁFICAS (MAX. TRÊS):} MARCIO FREITAS, C. Sistemas Embarcados na Automação Industrial. SILVEIRA, L. Q. LIMA, W. Um breve histórico conceitual da Automação Industrial e Redes para Automação Industrial. BECKER, LB. PEREIRA, CE. From design to implementation: tool support for the development of object-oriented distributed real-time systems.

%%%%%%%%%%% FIM DO RESUMO %%%%%%%%%%%
%%%%%%%%%%% O RESUMO DEVE TER NO MÁXIMO DUAS PÁGINAS %%%%%%%%%%%


\end{document}


